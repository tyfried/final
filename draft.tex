\documentclass{article}
\usepackage{amsmath}
\usepackage{amssymb}
\usepackage{amsthm}

%\usepackage{graphicx}
%\usepackage{caption}
%\usepackage{epstopdf}
%\usepackage{float}


\newtheorem{theorem}{Theorem}
\numberwithin{theorem}{subsection}




\title{Exploring Bounds}
\author{Tyler Friedman}


\begin{document}
\maketitle

\section{Introduction}

An interesting problem in coding theory is how to determine the maximum possible
number of codewords in a code.  For an $n$-dimensional code, we are interested in 
the maximum number of codewords such that the code has weight $d$.  
Formally, we use the notation $A_q(n,d)$
and $B_q(n,d)$, defined as the maximum of number of codewords in a code over
$\mathbb{F}_q$ of length $n$ and minimum distance $d$ for an arbitrary (linear
on non-linear) code and linear code, respectively.  It is not always possible to In lecture, we have
considered upper bounds---Sphere Packing, Singleton, and Greismer---as well as
lower bounds---Gilbert and Varshamov---on these values.  The purpose of this
paper is to consider other well-known bounds in the literature.

This paper is broken up into $x$ sections.


\section{Asymptotic Bounds}
We now wish to consider these $A_q(n,d)$ values as n goes to infinity.  To do so, we must more formally define two terms.

In class, we have considered the \textit{rate} of a linear code, $k/n$, as one measure of the goodness of a code.  That is,
the rate tells us how much information relative to redundancy that our codewords provide.  The concept of rate can be generalized
to non-linear codes as well.  For a possibly nonlinear code over $\mathbb{F}_q$ with $M$ codewords, the \textit{rate} is defined to be $n^{-1} \log_q {M}$.  Notice that
for an $[n,k,d]$ linear code, $M = q^k$ and hence the rate is $k/n$ as we expect.

A second notion of goodness that we have discussed, but I think not formally defined, is the \textit{relative distance} of a code.  For a linear or nonlinear code of length n
has minimum distance d, this value is the ratio $d/n$. 

Consequently, for our asymptotic bounds, we are interested in the largest possible rate for a family of codes over $\mathbb{F}_q$ of lengths going to infinity with relative 
distances approaching some constant $\delta$.  In other words, we consider the equation:

\begin{equation}
\alpha_{q}(\delta) = \limsup_{n \to \infty} n^{-1} \log_q A_q(n,\delta n)
\end{equation}


[What does this imply?]

\subsection{Asymptotic Singleton Bound}

Recall the Singleton Bound from lecture:

\begin{theorem}
For $d \le n$, $A_q(n,d) \le q^{n-d+1}$.
\end{theorem}

[explain what this means, what are the implications]


\begin{theorem}
If $0 \le \delta \le 1$, then $\alpha_q(\delta) \le 1 - \delta$.
\end{theorem} 
This theorem follows directly from the Singleton bound.

\begin{proof}
\begin{equation} 
\begin{split}
\alpha_{q}(\delta) & = \limsup_{n \to \infty} n^{-1} \log_q A_q(n,\delta n) \\
& \le \limsup_{n \to \infty} n^{-1} \log_q q^{n-\delta n+1} \\
& \le \limsup_{n \to \infty} \frac{n-\delta n+1}{n} \\
& \le 1 - \delta
\end{split}
\end{equation}
\end{proof}

\subsection{Asymptotic Plotkin Bound}



%First explore linear programming bound
Asymptotic singleton, include 2 examples
Asymptotic plotkin, some examples
draw graph
Asymptotic hamming
	define hilbert, prove 135,136, define vq(n,a),
	black box 2.10.3, prove 2.10.5


Table of common$a_q(n,d)$ values
Recall basic theory: thm 2.1.2, 2.1.6

Sections
 
Plotkin
talk about importance
define it
prove it
work out some examples

Elias
talk about importance
define it
prove two lemmas that do the heavy lifting
easy proof to finish it off
work out some examples

Linear Programming Bound
talk about importance
define it
prove it
work out some examples


Note existence of two lower bounds, say we talked about them in class







\end{document}
